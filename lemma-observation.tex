\documentclass[a4paper]{article}
\usepackage[english]{babel}
\usepackage[utf8x]{inputenc}
\usepackage[OT4]{fontenc}
\usepackage{amsthm,amssymb,amsmath,marginnote}

\DeclareMathOperator \Opt {\textbf{OPT}}
\DeclareMathOperator \Fazowy {\textbf{B}}

\newcommand \opt {\ensuremath{\Opt}}
\newcommand \fazowy {\ensuremath{\Fazowy}}

\begin{document}

\section{So-called Lemma 1.1}

Let $x_\text{i}$ and $x_\text{f}$ respectively be the location of \opt{} at the beginning and the end of the phase, and $x_\text{min}$ and $x_\text{max}$ respectively be the leftmost and rightmost points visited by \opt{} during the phase.

Assume that the location of the algorithm \fazowy{} at the beginning of the phase is $x_\text{i}$ as well, and its buffer contains at most $k$ requests.\marginnote{Do we need the property, że bufor \fazowy{} $\subseteq$ bufor \opt{}? Bo w razie czego można to uzupełnić} Then, \fazowy{} waits until the end of phase, during which $ck/2$ new requests arrive, so its buffer is not violated. Then it moves to $x_\text{min}$ or $x_\text{max}$, whichever of these points \opt{} has visited first during the phase, then it moves to the other of these two, clearing all the requests in the interval, and finally moves to $x_\text{f}$.

The requests cleared by \fazowy{} in the phase are all the requests in the interval $[x_\text{min},x_\text{max}]$ while \opt{} has possibly cleared only some of them. Nonetheless \opt{} has cleared at least $ck/2$ requests so that its buffer is not violated, thus \fazowy{} has also cleared at least that many requests, and its buffer at the end of the phase contains at most $k$ requests. Also, at the end of the phase \fazowy{} is at $x_\text{f}$, the same location as \opt, and therefore it may proceed to the next phase. Finally, the distance covered by \fazowy{} is not larger than the distance covered by \opt{}, and thus
\[ \fazowy(\sigma) \leq \opt(\sigma) \]

\section{So-called Observation 1.2}

First note that if the bucket arrangement $\{b_i\}$ centered at $p$ fulfills the bucket invariant, then any bucket is at most four times as large as the distance from $p$ to its beginning. Indeed, for $m$-th bucket this distance equals to
\[ \sum_{i=0}^{m-1}|b_i| \geq 1+1+\sum_{i=2}^{m-1}\tfrac12\cdot2^{i-1}
                            = 1+2^{m-2} = 1+\tfrac14\cdot2\cdot2^{m-1}
                         \geq 1+\tfrac14|b_m| \]
so $|b_m|<4\sum_{i=0}^{m-1}|b_i|$.

Let $m$ now be the index of the bucket containing $x_\text{max}$ defined as above. Considering only buckets to the right of $p$, we see that
\[ \sum_{i=1}^m|b_i^R| = \sum_{i=1}^{m-1}|b_i^R|+|b_m^R|
                       < (x_\text{max}-p)+4(x_\text{max}-p) \]
Similarly, considering only buckets to the left of $p$ and repeating the argument for $x_\text{min}\in b_{m'}^L$ we get $\sum_{i=1}^{m'}|b_i^L| < 5(p-x_\text{min})$. Putting those together, we obtain
\[ \sum_{b_i\in A(f)}|b_i| = \sum_{i=1}^{m}|b_i^R| + 1 + \sum_{i=1}^{m'}|b_i^L|
                           < 5(x_\text{max}-p)     + 1 + 5(p-x_\text{min}) \]
and thus, $5\opt{}(f)\geq5(x_\text{max}-x_\text{min})\geq\sum_{b_i\in A(f)}|b_i|$.


\thispagestyle{empty}
\end{document}
