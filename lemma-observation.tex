\documentclass[a4paper]{article}
\usepackage[english]{babel}
\usepackage[utf8x]{inputenc}
\usepackage[OT4]{fontenc}
\usepackage{amsthm,amssymb,amsmath,marginnote}

\DeclareMathOperator \Opt {\textbf{OPT}}
\DeclareMathOperator \Fazowy {\textbf{B}}

\newcommand \opt {\ensuremath{\Opt}}
\newcommand \fazowy {\ensuremath{\Fazowy}}

\begin{document}

\section{So-called Lemma 1.1}

Let $x_\text{i}$ and $x_\text{f}$ respectively be the location of \opt{} at the 
beginning and the end of the phase, and $x_\text{min}$ and $x_\text{max}$ 
respectively be the leftmost and rightmost points visited by \opt{} during the 
phase.

The algorithm \fazowy{} will maintain the following invariants: that its 
location at the beginning of the phase is $x_\text{i}$ as well, and the 
requests in its buffer are a subset of requests in the buffer of \opt. 
%Assume that the location of the algorithm \fazowy{} at the beginning of the 
%phase is $x_\text{i}$ as well, and its buffer contains at most $k$ requests.
%\marginnote{Do we need the property, że bufor \fazowy{} $\subseteq$ bufor \opt
%{}? Bo w razie czego można to uzupełnić} Then,
\fazowy{} waits until the end of phase, during which $ck/2$ new requests 
arrive, so its buffer is not violated. Then it moves to $x_\text{min}$ or 
$x_\text{max}$, whichever of these points \opt{} has visited first during the 
phase, then it moves to the other of these two, clearing all the requests in 
the interval, and finally moves to $x_\text{f}$.

The requests cleared by \fazowy{} in the phase are all the requests in the 
interval $[x_\text{min},x_\text{max}]$ while \opt{} has possibly cleared only 
some of them.
%Nonetheless \opt{} has cleared at least $ck/2$ requests so that its buffer is 
%not violated, thus \fazowy{} has also cleared at least that many requests, and 
%its buffer at the end of the phase contains at most $k$ requests.
Therefore the requests cleared by \opt{} are a subset of those cleared by 
\fazowy, and consequently the buffer of \fazowy{} at the end of the phase is a 
subset of the buffer of \opt. Also at this moment, which is the beginning of a 
new phase, \fazowy{} is at $x_\text{f}$, the same location as \opt, so both 
invariants of \fazowy{} are maintained and it may proceed to the next phase. 
Finally, the distance covered by \fazowy{} is not larger than the distance 
covered by \opt{}, and thus
\[ \fazowy(\sigma) \leq \opt(\sigma) \]

\section{So-called Observation 1.2}

First note that if the bucket arrangement $\{b_i\}$ fulfills the bucket 
invariant, then any bucket is at most four times as large as the distance from 
the origin to its beginning.\marginnote{Cichaczem wprowadzam nowe pojęcie --- 
orydżyn arendżmentu bakietów.} Indeed, for $m$-th bucket this distance equals to

\[ \sum_{i=0}^{m-1}|b_i| \geq 1+1+\sum_{i=2}^{m-1}\tfrac12\cdot2^{i-1}
                            = 1+2^{m-2} = 1+\tfrac14\cdot2\cdot2^{m-1}
                         \geq 1+\tfrac14|b_m| \]
so $|b_m|<4\sum_{i=0}^{m-1}|b_i|$.

Let $d_\text{R}$ be the distance from the origin to the rightmost point visited 
by \opt{} in a phase and $m$ be the index of the bucket containing that point. 
The total size of the right-hand-side buckets visited by \opt{} is thus
\[ \sum_{i=1}^m|b_i^\text{R}| = \sum_{i=1}^{m-1}|b_i^\text{R}|+|b_m^\text{R}|
                              < 5\sum_{i=1}^{m-1}|b_i^\text{R}|
                           \leq 5d_\text{R} \]
Similarly, for $d_\text{L}$ being the distance from the origin to the leftmost 
point visited by \opt{} and $m'$ the index of the bucket containing it, we get 
$\sum_{i=1}^{m'}|b_i^\text{L}| < 5d_\text{L}$. Putting those together, we obtain
\[ \sum_{b_i\in A(f)}|b_i| = \sum_{i=1}^{m}|b_i^\text{R}| + 1
                           + \sum_{i=1}^{m'}|b_i^\text{L}|
                           < 5d_\text{R} + 1 + 5d_\text{L} \]
and thus, $5\opt{}(f)\geq5(d_\text{L}+d_\text{R})\geq\sum_{b_i\in A(f)}|b_i|$.

\thispagestyle{empty}
\end{document}
