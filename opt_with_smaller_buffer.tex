\section{OPT with smaller buffer}
We have $\opt_k$ with buffer of size $k$. We divide input sequence $\sigma$ into
phases. They contain $\frac{c \cdot k}{2}$ consecutive requests each, are
disjoint and form a coverage of $\sigma$.

\begin{lemma}\label{lem:algorithm_B}
There exists an offline algorithm \fazowy{} for sorting buffer problem, which
has cache of size $(1 + \frac{c}{2}) \cdot k$, makes actions only after every
phase and makes the same moves as \opt{}. For any input sequence $\sigma$, its
cost is $\fazowy(\sigma) = O(1) \cdot \opt_k(\sigma)$.
\end{lemma}

\begin{proof}
Let $x_\text{i}$ and $x_\text{f}$ respectively be the location of \opt{} at 
the beginning and the end of the phase, and $x_\text{min}$ and $x_\text{max}$ 
respectively be the leftmost and rightmost points visited by \opt{} during the 
phase.

The algorithm \fazowy{} will maintain the following invariants: that its 
location at the beginning of the phase is $x_\text{i}$ as well, and the 
requests in its buffer are a subset of requests in the buffer of \opt. 
%Assume that the location of the algorithm \fazowy{} at the beginning of the 
%phase is $x_\text{i}$ as well, and its buffer contains at most $k$ requests.
%\marginnote{Do we need the property, że bufor \fazowy{} $\subseteq$ bufor \opt
%{}? Bo w razie czego można to uzupełnić} Then,
\fazowy{} waits until the end of phase, during which $ck/2$ new requests 
arrive, so its buffer is not violated. Then it moves to $x_\text{min}$ or 
$x_\text{max}$, whichever of these points \opt{} has visited first during the 
phase, then it moves to the other of these two, clearing all the requests in 
the interval, and finally moves to $x_\text{f}$.

The requests cleared by \fazowy{} in the phase are all the requests in the 
interval $[x_\text{min},x_\text{max}]$ while \opt{} has possibly cleared only 
some of them.
%Nonetheless \opt{} has cleared at least $ck/2$ requests so that its buffer is 
%not violated, thus \fazowy{} has also cleared at least that many requests, and 
%its buffer at the end of the phase contains at most $k$ requests.
Therefore the requests cleared by \opt{} are a subset of those cleared by 
\fazowy, and consequently the buffer of \fazowy{} at the end of the phase is a 
subset of the buffer of \opt. Also at this moment, which is the beginning of a 
new phase, \fazowy{} is at $x_\text{f}$, the same location as \opt, so both 
invariants of \fazowy{} are maintained and it may proceed to the next phase. 
Finally, the distance covered by \fazowy{} is not larger than the distance 
covered by \opt{}, and thus
\[ \fazowy(\sigma) \leq \opt(\sigma). \qedhere \]
\end{proof}

Let $b(f)$ denote the position of \fazowy{} at the end of phase $f$. During
phase $f$, \fazowy{} moves from point $b(f-1)$ to point $b(f)$ and it clears
some interval $B(f)$ that includes interval $[\min(b(f-1), b(f)), \max(b(f-1),
b(f))]$.

\begin{definition}
A \emph{bucket arrangement centered in $p$} is a partition of the line into
consecutive segments $\{b_i^\text{L},b_0,b_i^\text{R}\}$ such that,
additionally, $b_0$ contains only point $p$, the \emph{origin} of the
arrangement. Assume $b_0^\text{L}=b_0^\text{R}=b_0$.

A bucket arrangement is \emph{regular} (or consists of regular buckets) if the
$i$-th buckets on either side ($b_i^\text{L},b_i^\text{R}$) have size exactly
$2^i$, except possibly for the leftmost and rightmost buckets, which may be
smaller. A bucket arrangement which is not regular is \emph{perturbed} (or
consists of perturbed buckets).

A bucket arrangement fulfills \emph{bucket invariant} if the size of
$b_i^\text{L},b_i^\text{R}$ deviates from the size of corresponding regular
bucket only by a constant factor, ie.\ $l\cdot2^i\leq|b_i|\leq
h\cdot2^i$,\marginnote{Set $h,l$.} except possibly for the leftmost and
rightmost buckets, for which only the upper bound needs to hold.
%
%\textbf{Regular buckets centered in $\mathbf{p}$}, denoted by $h^r_i$ and
%$h^l_i$, $i \in \mathbb{N}$, are defined as follows: $h^r_i$, consists of
%points on the right hand side of $p$, whose distance from $p$ is more or equal
%to $2^{|i| - 1}$ but less then $2^i$. For $h^l_i$ points are taken from the
%left side of $p$ and defined the same way. For $i=0$, $h^r_0 = h^l_0 = h_0$ we
%have special bucket. It contains only point $p$.
%
%\textbf{Distorted buckets centered in $\mathbf{p}$}, $h'^r_i$ and $h'^l_i$,
%differ from regular buckets in size. Specifically, $i$-th bucket's size can
%differ from corresponding $i$-th regular bucket size at most by factor $2$, so
%$\frac{1}{2} \leq \frac{|h'^r_i|}{|h^r_i|} \leq 2$ (similar for 'left'
%buckets). From now on this inequality will be called \textbf{bucket invariant}.
\end{definition}

We almost always consider bucket arrangements that fulfill the bucket invariant.

\begin{observe}
Let $p$ be the point of $\opt_k$ at the start of phase $f$. Let $A(f)$ be the
set of distorted buckets that were visited by $\opt_k$ during phase $f$ (the
rightmost and leftmost buckets are not, in general, traversed throughout). Sum
of sets in $A(j)$ forms continuous section. Moreover,
\[\opt_k(f) \geq O(1) \cdot \sum_{b \in A(f)} |b|.\]
\end{observe}

\begin{proof}
First note that if the bucket arrangement $\{b_i\}$ fulfills the bucket
invariant, then any bucket is at most four times as large as the distance from
the origin to its beginning. Indeed, for $m$-th bucket on, say, the right this
distance equals to
\[ \sum_{i=0}^{m-1}|b_i^\text{R}| \geq 1 +  l\sum_{i=1}^{m-1}2^i = 1+l(2^m-2)
                                  \geq 1 - 2l + l\cdot\frac{|b_m^\text{R}|}h \]
so $|b_m^\text{R}|\leq\frac hl\sum_{i=0}^{m-1}|b_i^\text{R}|$.\marginnote{Needs
slight fixing if $l>\frac12$.}

Let $d_\text{R}$ be the distance from the origin to the rightmost point visited 
by \opt{} in a phase and $m$ be the index of the bucket containing that point. 
The total size of the right-hand-side buckets visited by \opt{} is thus
\[ \sum_{i=0}^m|b_i^\text{R}| = \sum_{i=0}^{m-1}|b_i^\text{R}|+|b_m^\text{R}|
                           \leq (\tfrac hl + 1)\sum_{i=0}^{m-1}|b_i^\text{R}|
                           \leq \tfrac{h+l}l\cdot d_\text{R} \]
Similarly, for $d_\text{L}$ being the distance from the origin to the leftmost 
point visited by \opt{} and $m'$ the index of the bucket containing it, we get 
$\sum_{i=0}^{m'}|b_i^\text{L}| \leq \tfrac{h+l}l\cdot d_\text{L}$. Putting those
together, we obtain
\[ \sum_{b\in A(f)}|b| \leq \sum_{i=0}^{m}|b_i^\text{R}|
                          + \sum_{i=0}^{m'}|b_i^\text{L}|
                       \leq \tfrac{h+l}l\cdot d_\text{R}
                          + \tfrac{h+l}l\cdot d_\text{L} \]
and thus, $\tfrac{h+l}l\opt{}(f)\geq\tfrac{h+l}l(d_\text{L}+d_\text{R})
\geq\sum_{b_i\in A(f)}|b_i|$.
\end{proof}

The algorithm for \fazowy{} is duplicating the $\opt_k$ moves during 
one phase. When $\opt_k$ changes its position, the buckets would not be 
distributed as described previously. We want to introduce algorithm 
\przesuw, which is a modification of algorithm \fazowy. Recall, $B(f)$ 
for phase $f$ is continuous interval. In phase $f$, instead of $B(f)$, 
algorithm \przesuw{} will clear buckets' set $C(F) \supseteq B(f)$ (these 
distorted buckets are centered in $b(f-1)$). It then moves to point $b(f)$ and 
reorganizes buckets in the interval $C(f)$, so they fulfill the invariant for 
distorted buckets centered in $b(f)$. 

\begin{lemma}
We can define $C(f)$ in such a way, that summing over all phases $\sum_{f - 
phase} |C(f)| = O(1) \cdot \sum_{f - phase} |B(f)|$. What is more, inside 
$C(f)$, \przesuw{} changes buckets, so they fulfill invariant after 
every change of \przesuw{} position. The positions of \przesuw{} are at the end 
of phase are the same as positions of \fazowy. Cost of bucket reorganization 
in all phases can be bounded by $O(1) \cdot \sum_{f-phase} C(f)$.
\end{lemma}

\begin{theorem}
For any input sequence $\sigma$, $\przesuw(\sigma)\leq O(\log k)\opt_k(\sigma)$.
\end{theorem}
