\section{OPT with smaller buffer}
We have $\mathrm{OPT_k}$ with buffer of size $k$. We divide input sequence 
$\sigma$ into phases. They 
contain $\frac{c \cdot k}{2}$ consecutive requests each, are disjoint and form 
a 
coverage of $\sigma$.
\begin{lemma}
Fix input sequence $\sigma$. There exists an offline algorithm \textbf{B} 
for sorting buffer problem, which has cache of size $(1 + 
\frac{c}{2}) \cdot k$, makes actions only after every phase and makes the same 
moves as \textbf{OPT}. Its cost is $\mathrm{B}(\sigma) = O(1) \cdot 
\mathrm{OPT_k}(\sigma)$.
\label{lem:algorithm_B}
\end{lemma}
Let $b(f)$ denote the position of $B$ at the end of phase $f$. During phase 
$f$, \textbf{B} moves from point $b(f-1)$ to point $b(f)$ and it clears some 
interval $B(f)$ that includes interval $[min(b(f-1), b(f)), max(b(f-1), 
b(f))]$.
Define \textit{pure buckets} $h^r_i$ and $h^l_i$, $i \in \mathbb{N}$, for a 
given point on line $p$. Pure bucket 
$h^r_i$, consists of points on the right hand side of $p$, whose 
distance 
from $p$ is more then $2^{|i| - 1}$ but less or equal then $2^i$. For $h^l_i$ 
we have similar definition, but the points are taken from the 
left side of $p$. For $i=0$, $h^r_0 = h^l_0 = h_0$ we have special bucket. It 
contains only point $p$. Now we define \textit{buckets} (not pure) 
$h'^r_i$ and $h'^l_i$ for point $p$. They are differ from pure buckets in a 
way, that $i$-th bucket's size can differ from corresponding $i$-th pure bucket 
size at most by factor $2$, so $\frac{1}{2} \leq \frac{|h'^r_i|}{|h^r_i|} \leq 
2$ (similar for 'left' buckets). From now on this inequality will be called 
\textit{bucket invariant}.
\begin{observe}
Let $p$ be the point of $\mathrm{OPT_k}$ at the start of phase $f$. Let $A(f)$ 
be 
the set of buckets $b_i$ that were visited by $\mathrm{OPT_k}$ during phase $f$ 
(the 
'largest index' buckets might have been visited only partially. Sum of sets in 
$A(j)$ 
forms continuous section. Moreover, $$\mathrm{OPT_k}(f) \geq \sum_{b_i \in 
A(f)} |b_i|.$$
\end{observe}
The algorithm for \textbf{B} is duplicating the $\mathrm{OPT_k}$ moves during 
one phase. When $\mathrm{OPT_k}$ changes its position, the buckets would not be 
distributed as described previously. We want to introduce algorithm 
\textbf{C} based on algorithm \textbf{B}. Recall, $B(f)$ for phase $f$ is 
continuous interval. In phase $f$, instead of $B(f)$, algorithm \textbf{C} will 
clear buckets' set $C(F) \supseteq B(f)$ (these buckets are defined with 
respect to $b(f-1)$). It then moves to point $b(f)$ and reorganizes buckets in 
the interval $C(f)$, so they fulfill the bucket invariant with respect to 
point $b(f)$. 
\begin{lemma}
We can define $C(f)$, that summing over all phases $\sum_{f - phase} |C(f)| = 
O(1) \cdot \sum_{f - phase} |B(f)|$. What is more, inside 
$C(f)$ \textbf{C} changes buckets, so they fulfill 
buckets' invariant after every change of 
\textbf{C} position in phase, which is the same as \textbf{B}'s position. Cost 
of bucket reorganization in all phases can be bounded 
by $O(1) \cdot \sum_{f-phase} C(f)$.
\end{lemma}
\begin{theorem}
 For any input sequence $\sigma$, $ \log(k) \cdot \mathrm{OPT_k}(\sigma) \leq
C(f)$.
\end{theorem}

