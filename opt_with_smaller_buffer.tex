\section{OPT with smaller buffer}
We have $\mathrm{OPT_k}$ with buffer of size $k$. We divide input sequence 
$\sigma$ into phases. They 
contain $\frac{c \cdot k}{2}$ consecutive requests each, are disjoint and form 
a 
coverage of $\sigma$.
\begin{lemma}
Fix input sequence $\sigma$. There exists an offline algorithm \textbf{B} 
for sorting buffer problem, which has cache of size $(1 + 
\frac{c}{2}) \cdot k$, makes actions only after every phase and makes the same 
moves as \textbf{OPT}. Its cost is $\mathrm{B}(\sigma) = O(1) \cdot 
\mathrm{OPT_k}(\sigma)$.
\label{lem:algorithm_B}
\end{lemma}
\begin{proof}
 Let $x_\text{i}$ and $x_\text{f}$ respectively be the location of \opt{} at 
the 
beginning and the end of the phase, and $x_\text{min}$ and $x_\text{max}$ 
respectively be the leftmost and rightmost points visited by \opt{} during the 
phase.

The algorithm \fazowy{} will maintain the following invariants: that its 
location at the beginning of the phase is $x_\text{i}$ as well, and the 
requests in its buffer are a subset of requests in the buffer of \opt. 
%Assume that the location of the algorithm \fazowy{} at the beginning of the 
%phase is $x_\text{i}$ as well, and its buffer contains at most $k$ requests.
%\marginnote{Do we need the property, że bufor \fazowy{} $\subseteq$ bufor \opt
%{}? Bo w razie czego można to uzupełnić} Then,
\fazowy{} waits until the end of phase, during which $ck/2$ new requests 
arrive, so its buffer is not violated. Then it moves to $x_\text{min}$ or 
$x_\text{max}$, whichever of these points \opt{} has visited first during the 
phase, then it moves to the other of these two, clearing all the requests in 
the interval, and finally moves to $x_\text{f}$.

The requests cleared by \fazowy{} in the phase are all the requests in the 
interval $[x_\text{min},x_\text{max}]$ while \opt{} has possibly cleared only 
some of them.
%Nonetheless \opt{} has cleared at least $ck/2$ requests so that its buffer is 
%not violated, thus \fazowy{} has also cleared at least that many requests, and 
%its buffer at the end of the phase contains at most $k$ requests.
Therefore the requests cleared by \opt{} are a subset of those cleared by 
\fazowy, and consequently the buffer of \fazowy{} at the end of the phase is a 
subset of the buffer of \opt. Also at this moment, which is the beginning of a 
new phase, \fazowy{} is at $x_\text{f}$, the same location as \opt, so both 
invariants of \fazowy{} are maintained and it may proceed to the next phase. 
Finally, the distance covered by \fazowy{} is not larger than the distance 
covered by \opt{}, and thus
\[ \fazowy(\sigma) \leq \opt(\sigma) \]
\end{proof}

Let $b(f)$ denote the position of $B$ at the end of phase $f$. During phase 
$f$, \textbf{B} moves from point $b(f-1)$ to point $b(f)$ and it clears some 
interval $B(f)$ that includes interval $[min(b(f-1), b(f)), max(b(f-1), 
b(f))]$.
\begin{definition}
\textbf{Regular buckets centered in $\mathbf{p}$}, denoted by $h^r_i$ and 
$h^l_i$, $i \in \mathbb{N}$,  are defined as follows: 
$h^r_i$, consists of points on the right hand side of $p$, whose 
distance 
from $p$ is more or equal to $2^{|i| - 1}$ but less then $2^i$. 
For $h^l_i$ points are taken from the left side of $p$ and defined the same 
way. For $i=0$, $h^r_0 = h^l_0 = h_0$ we have special bucket. It contains only 
point $p$.

\textbf{Distorted buckets centered in $\mathbf{p}$}, $h'^r_i$ and $h'^l_i$,  
differ from regular buckets in size. Specifically, $i$-th bucket's size can 
differ from corresponding $i$-th regular bucket 
size at most by factor $2$, so $\frac{1}{2} \leq \frac{|h'^r_i|}{|h^r_i|} \leq 
2$ (similar for 'left' buckets). From now on this inequality will be called 
\textbf{bucket invariant}.
\end{definition}
\begin{observe}
Let $p$ be the point of $\mathrm{OPT_k}$ at the start of phase $f$. Let $A(f)$ 
be 
the set of distorted buckets that were visited by $\mathrm{OPT_k}$ during 
phase $f$ 
(the 
largest index right and left bucket might have been visited only partially). 
Sum of sets in 
$A(j)$ 
forms continuous section. Moreover, $$\mathrm{OPT_k}(f) \geq O(1) 
\cdot \sum_{b \in 
A(f)} |b|.$$
\end{observe}
\begin{proof}
 First note that if the bucket arrangement $\{b_i\}$ fulfills the bucket 
invariant, then any bucket is at most four times as large as the distance from 
the origin to its beginning.\marginnote{Cichaczem wprowadzam nowe pojęcie --- 
orydżyn arendżmentu bakietów.} Indeed, for $m$-th bucket this distance equals to

\[ \sum_{i=0}^{m-1}|b_i| \geq 1+1+\sum_{i=2}^{m-1}\tfrac12\cdot2^{i-1}
                            = 1+2^{m-2} = 1+\tfrac14\cdot2\cdot2^{m-1}
                         \geq 1+\tfrac14|b_m| \]
so $|b_m|<4\sum_{i=0}^{m-1}|b_i|$.

Let $d_\text{R}$ be the distance from the origin to the rightmost point visited 
by \opt{} in a phase and $m$ be the index of the bucket containing that point. 
The total size of the right-hand-side buckets visited by \opt{} is thus
\[ \sum_{i=1}^m|b_i^\text{R}| = \sum_{i=1}^{m-1}|b_i^\text{R}|+|b_m^\text{R}|
                              < 5\sum_{i=1}^{m-1}|b_i^\text{R}|
                           \leq 5d_\text{R} \]
Similarly, for $d_\text{L}$ being the distance from the origin to the leftmost 
point visited by \opt{} and $m'$ the index of the bucket containing it, we get 
$\sum_{i=1}^{m'}|b_i^\text{L}| < 5d_\text{L}$. Putting those together, we obtain
\[ \sum_{b_i\in A(f)}|b_i| = \sum_{i=1}^{m}|b_i^\text{R}| + 1
                           + \sum_{i=1}^{m'}|b_i^\text{L}|
                           < 5d_\text{R} + 1 + 5d_\text{L} \]
and thus, $5\opt{}(f)\geq5(d_\text{L}+d_\text{R})\geq\sum_{b_i\in A(f)}|b_i|$.
\end{proof}

The algorithm for \textbf{B} is duplicating the $\mathrm{OPT_k}$ moves during 
one phase. When $\mathrm{OPT_k}$ changes its position, the buckets would not be 
distributed as described previously. We want to introduce algorithm 
\textbf{C}, which will a modification of algorithm \textbf{B}. Recall, $B(f)$ 
for phase $f$ is continuous interval. In phase $f$, instead of $B(f)$, 
algorithm \textbf{C} will clear buckets' set $C(F) \supseteq B(f)$ (these 
distorted buckets are centered in $b(f-1)$). It then moves to point $b(f)$ and 
reorganizes buckets in the interval $C(f)$, so they fulfill the invariant for 
distorted buckets centered in $b(f)$. 
\begin{lemma}
We can define $C(f)$ in such a way, that summing over all phases $\sum_{f - 
phase} |C(f)| = O(1) \cdot \sum_{f - phase} |B(f)|$. What is more, inside 
$C(f)$, \textbf{C} changes buckets, so they fulfill invariant after 
every change of \textbf{C} position. The positions of \textbf{C} are at the end 
of phase are the same as \textbf{B}'s positions. Cost of bucket reorganization 
in all phases can be bounded by $O(1) \cdot \sum_{f-phase} C(f)$.
\end{lemma}
\begin{theorem}
For any input sequence $\sigma$, $ \log(k) \cdot \mathrm{OPT_k}(\sigma) \leq
C(f)$.
\end{theorem}

